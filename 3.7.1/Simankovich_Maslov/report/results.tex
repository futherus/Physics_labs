\section*{Выводы}

В работе исследовался скин-эффект в медном полом цилиндре.

Была экспериментально подтверждена зависимость модуля магнитного поля $H_1$ внутри цилиндра от величины внешнего магнитного поля $H_0$ при низких частотах 

$$
|H_1| = \frac{|H_0|}{\sqrt{1 + \left( \frac{ah}{\Lambda} \right)^2}} = \frac{1}{\sqrt{1 + \frac{1}{4} \left( a h \sigma \mu_0 \omega \right)^2}}
$$
где $a$ -- радиус цилиндра, $h$ -- толщина стенок цилиндра, $\Lambda = \sqrt{\frac{2}{\mu_0 \sigma \omega}}$ -- толщина скин-слоя.

Была определена проводимость медного цилиндра в области низких частот $$
\sigma = 2.17 \cdot 10^7 \; \frac{См}{м}
$$

Экспериментально была подтверждена теоретическая зависимость сдвига фазы между магнитным полем внутри цилиндра $H_1$ и вне $H_0$ в области низких частот:
$$
\tg \psi = \left( \frac{ah}{\Lambda} \right)^2.
$$

Была определена проводимость медного цилиндра вторым способом в области низких частот:
$$
\sigma = 2.41 \cdot 10^7 \; \frac{См}{м}
$$

Экспериментально была подтверждена теоретическая зависимость сдвига фазы между магнитным полем внутри цилиндра $H_1$ и вне $H_0$ в области высоких частот:
$$
\psi = \frac{\pi}{4} + \frac{h}{\Lambda}
$$

Была определена проводимость в области высоких частот:
$$
\sigma = 3.91 \cdot 10^7 \; \frac{См}{\text{м}}
$$

С помощью $RLC$-метра была определена проводимость цилиндра:
$$
\sigma = 1.99 \cdot 10^7 \; \frac{\text{См}}{\text{м}}
$$

