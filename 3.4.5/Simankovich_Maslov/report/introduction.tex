\section*{Введение}

Из опытов известно, что вещество может реагировать на внешнее магнитное поле. Ферромагнетики проявляют сильное взаимодействие и усиливают внешнее магнитное поле в $10^3 \div 10^4$ раз. Для ферромагнетиков характерно существование \textit{петель гистерезиса} -- нелинейности зависимости $B(H)$. Данное явление описывается теоретически с помощью уравнений квантовой механики. Альтернативным способом описания является эмпирическая теория ферромагнетизма Вейса.

Данная работа была проведена в рамках учебного исследовательского курса в Московском физико-техническом институте. Целью работы является подтверждение существования петель гистерезиса для образцов феррита (1000 нн), пермаллоя (Fe-Ni, нп50), кремнистого железа (Fe-Si).

