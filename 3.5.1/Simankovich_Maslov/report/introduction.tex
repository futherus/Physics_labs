\section*{Введение}

\textit{Плазмой} называется частично или полностью ионизованный квазинейтральный газ, занимающий настолько большой объём, что в нём не происходит сколько-нибудь заметного нарушения квазинейтральности из-за тепловых флуктуаций. 

Из-за большого количества заряженных частиц, плазма сильно взаимодействует с внешними электрическими и магнитными полями и обладает высокой электропроводностью. 

\textit{Квазинейтральность} означает, что плотности положительных и отрицательных зарядов одинаковы. Если бы в плазме присутствовали нескомпенсированные заряды, то равновесие было бы невозможно. В плазме возникли бы сильные электромагнитные поля, приводящие заряды в движение, и устанавливающие равновесие. Поэтому в плазме плотности положительных и отрицательных зарядов должны практически совпадать. Отклонение от квазинейтральности возможно лишь на микроскопических расстояниях из-за тепловых флуктуаций.

Плазма отличается от нейтрального идеального газа тем, что на микроуровне частицы взаимодействуют между собой не только во время столкновений, но и во время движения посредством электромагнитных сил. То есть для плазмы характерны \textit{коллективные взаимодействия}, когда силы, действующие на частицу формируются группой частиц, а не в результате прямых парных взаимодействий. 

Плазма называется \textit{низкотемпературной}, если её температура  $T < 10^4 \; К$. Плазма называется \textit{высокотемпературной}, если $T > 10^6 \; К$. Низкотемпературная плазма применяется для проведения химических реакций в газах, в методах плазменного травления при создании интегральных микросхем. В природе плазма встречается в ионосфере Земли. Из высокотемпературной полностью ионизованной плазмы состоят Солнце и горячие звезды, в которых происходят реакции термоядерного синтеза.