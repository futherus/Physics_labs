\section*{Экспериментальные результаты}

\subsection*{Вольт-амперная характеристика разряда}

Определим напряжение зажигания разряда. Для этого будем плавно поднимать напряжение ВИП. В момент зажигания разряда $U_{\text{заж}} = 2140$ В.

Проведем измерения ВАХ газоразрядной трубки с помощью амперметра $A_1$ и вольтметра $V_1$. Построим график $U_p(I_p)$ \figref{fig:upip}.

\begin{figure}[H]
	\centering
	\includegraphics[width=0.8\linewidth]{../gen/U_p_I_p.pdf}
	\caption{ВАХ разрядной трубки}
	\label{fig:upip}
\end{figure}

Определим также максимальное дифференциальное сопротивление разряда $R_{\text{дифф}}$.
$$ R_{\text{дифф}} = \frac{dU}{dI} = -44.8 \; \text{кОм}.$$

\subsection*{Зондовые характеристики}

Проведем измерения зондовых характеристик разряда при различных разрядных токах $I_p$. Измерения будем проводить при обоих полярностях зондов.

\begin{figure}[H]
	\centering
	\includegraphics{../gen/Iz_Uz.pdf}
	\caption{Зондовые характеристики}
	\label{fig:izuz}
\end{figure}

Вычислим значение температуры электронов $T_e$.

$$ kT_e = \frac{1}{2} \frac{e I_n}{\frac{dI}{dU}} \Rightarrow kT_e = \Delta U / 2 \quad [\text{СИ}] $$
где $\Delta U$ -- абсциссы точек, помеченных "крестиками".

Воспользуемся формулой Бома чтобы определить концентрацию электронов $n_e$. Будем считать, что $n_e = n_i$ -- концентрации ионов.

$$ I_n = 0.4 n_e e S \sqrt{\frac{2k T_e}{m_i}} \Rightarrow n_e = \frac{I_n}{0.4 e S} \sqrt{\frac{m_i}{2 k T_e}} \quad [\text{СИ}], $$
где $S = \pi d l = 3.26 \; \text{мм}^2$ -- площадь зонда, $m_i = 22 \cdot 1.66 \cdot 10^{-27}$ -- масса иона неона.

Также рассчитаем плазменную (ленгмюровскую) частоту $\omega_p$:
$$ \omega_p = \sqrt{\frac{4\pi n_e e^2}{m_e}} \quad [\text{СГС}],$$
где $m_e$ -- масса электрона.

Определим характерные длины плазмы -- поляризационную $r_{De}$ и дебаевский радиус экранирования $r_D$.
$$ r_{De} = \sqrt{\frac{kT_e}{4\pi n_e e^2}} \qquad r_D = \sqrt{\frac{kT_i}{4\pi n_e e^2}}  \quad [\text{СГС}],$$
где $T_i \approx 300$ K -- температура ионов.

Оценим среднее число ионов в характерном объеме плазмы -- дебаевской сфере.
$$ N_d = \frac{4}{3} \pi r_d^3 n_i.$$

Давление в трубке оценивается значением $P \approx 2 \; \text{торр}$. Тогда можно оценить степень ионизации плазмы $\alpha = n_i/n$, где $n$ -- концентрация частиц в трубке. Концентрацию $n$ определим из следующего соотношения:
$$ P = n k T_i.$$

Построим графики зависимостей температуры электронов $T_e(I_p)$ и концентрации электронов $n_e(I_p)$, где $I_p$ -- разрядный ток.

\begin{figure}[H]
	\centering
	\begin{minipage}[c]{.5\textwidth}
		\centering
		\includegraphics[width=0.9\linewidth]{"../gen/T_e_I_p.pdf"}
		\caption{$T_e$ от тока в разряде}
		\label{img:teip}
	\end{minipage}%
	\begin{minipage}[c]{.5\textwidth}
		\centering
		\includegraphics[width=0.9\linewidth]{"../gen/n_e_I_p.pdf"}
		\caption{$n_e$ от тока в разряде}
		\label{img:neip}
	\end{minipage}
\end{figure}


